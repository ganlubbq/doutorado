%  SparseArch.tex

\documentclass[11pt,jeep]{book}
\pagestyle{headings}

\begin{document}

\input SparseMacros.tex

%=========================================

\title{\Huge SparseLab Architecture}
\author{
David Donoho, Victoria Stodden, and Yaakov Tsaig\\
Stanford University
\thanks{{\bf Acknowledgment of Support.}
This work was partially supported by NSF DMS-05-05303 and by other
sponsors.} }
\date{Version \WLVersion\\ \WLDate}

\maketitle

%\tableofcontents

%=============================================
%                Chapter 1
%=============================================
\chapter[Introduction]{Introduction}
\label{wlarch}

{\bf Changes and Enhancements for Release 2.0}: 4 papers have been added to Sparselab 2.0: "Fast
Solution of l1-norm Minimization Problems When the Solutions May be Sparse"; "Why Simple Shrinkage
is Still Relevant For Redundant Representations"; "Stable Recovery of Sparse Overcomplete
Representations in the Presence of Noise"; "On the Stability of Basis Pursuit in the Presence of
Noise."

This document describes the architecture of \WaveLab\ version \WLVersion.  It is designed for users
who already have had day-to-day interaction with the package and now need specific details about
the architecture of the package, for example to modify components for their own research.

For an introduction to \WaveLab\ at an elementary level, see {\it About SparseLab}. This document
may be accessed via WWW  through the \WaveLab\ Home Page: \WLWEB.

Before beginning, we mention the main components of the \WaveLab\ package, to standardize
terminology. First, there are the basic ``system components'':

\begin{itemize}
\item[1.] {\it Source.}  There is source code, in \Matlab, \TeX, Perl.

\item[2.] {\it Build.}  The source code is assembled into a standard release.
The current release is \WLVersion.

\item[3.] {\it Archives.} Compressed archives of the standard release available
for three platforms, Mac, Unix and PC, which users can download and install on
their machines.

\item[4.] {\it Web Documents.} A web home page (which can be viewed using any
web browser) and a series of postscript and pdf files which explain
what \WaveLab\ is and how to get it.  The URL is \WLWEB.
\end {itemize}

Next there are the  basic ``user components'' of an installed system:

\begin {itemize}
\item[1.] {\it \WaveLab\ Main Directory.}  A subdirectory /\WLName\/ of the {\tt
Matlab/work} directory, containing the currently released version of
\WaveLab\ software, datasets and documentation.

\item[2.] {\it Papers.} A directory \PaperDir\ in /\WLName\/ containing scripts
reproducing figures in various articles and technical reports.

\item[3.] {\it Examples.} A directory \WorkoutDir\ in /\WLName\
containing pedagogical examples that exercise various aspects of
\WaveLab.

\item[4.] {\it Solvers.} A directory /Sparselab100/Solvers/ in /\WLName\/ containing
the various solver engines of SparseLab.

\item[5.] {\it Documentation.} Both pdf and
Postscript files available by WWW access.

\item[6.] {\it Datasets.} The largest are included as separate downloads:
Sparselabvers\_DataSupplementExtCS and Sparselabvers\_DataSupplementStOMP, where vers is replaced
by the current SparseLab version. Numerical and image data used to illustrate various aspects of
sparse analysis by the scripts and workouts.
\end{itemize}

The following document describes all these various components from a
systems-level point of view. An individual needing to modify \WaveLab\ or add to
it would be interested in this information.

%=============================================
%                Chapter 2
%=============================================
\chapter[Papers]{Papers}
\label{scsec}

We briefly describe the contents and architecture of the
\PaperDir\ subdirectory of \WaveLab.

%-------------------------------------------------------------------------------
\section[Script Philosophy]{Script Philosophy}
\label{scphilo} The makeup of \PaperDir\ is the whole raison d'\^etre of the \WaveLab\ package. The
idea is that, when doing research in a computational science, one works to develop reproducible
knowledge about the results of computational experiments. The {\tt /Papers} directory is the end
product of such an effort. It makes available to researchers around the world, via the Internet,
the computations that produced figures which have been published in hardcopy form as technical
reports at Stanford University and in forthcoming journal articles. Other researchers can obtain
the \Matlab\ code which generated these figures, and can reproduce the calculations that underly
the figures.  They can, if they wish, modify the calculations by editing the underlying \Matlab\
code. They can use the algorithms on other datasets. They can try their own favorite methods on the
same datasets.

Our idea is that, when doing research, long before we write an article, we prepare ourselves with
the thought that {\it what we do on the computer will ultimately be made available to others, for
their inspection, modification, re-use and criticism}. This implies several things.  First, that
the work product which we are aiming to create will be a subdirectory of \WaveLab\ containing a
series of scripts that will generate, from scratch, all the figures of the corresponding article.
Second, that our work product is {\it not} the printed figures that go into the article, but the
underlying algorithms and code which generate those figures, and which will be made available to
others. Thus, it is no good to print a hardcopy of a figure that we see on the screen and save that
for photocopying into a final version of the paper.  Once we are happy with a figure we see on the
screen, we must save the code that generated the figure, and then edit the code to make it part of
a system that automatically reproduces all the figures of an article.

The philosophy we are adopting can be traced to Jon Claerbout and Martin Karrenbach's article {\it
Electronic Documents Give Reproducible Research New Meaning} ({\tt http://sepwww.stanford.edu}).
We especially like a thought of theirs which we paraphrase as follows:\\
\begin{quote} {\it \noindent A traditionally published article is not the end product of scholarship; it is the advertisement
for the scholarship. The working software environment that produced the figures in the article is
the actual end product of the scholarship.} \end{quote}

To work in accordance with the philosophy, we must adopt a discipline of how we structure our
computational experiments in \Matlab. A benefit of this discipline is, hopefully, to avoid the
sloppiness and errors that are ubiquitous in computational science.

%-------------------------------------------------------------------------------
\section[Script Architecture]{Script Architecture}
\label{scarch}

The architecture of the {\tt /Papers} directory is as follows. At present, it
contains these subdirectories, recreating figures in published articles:
\begin{verbatim}
  ExtCSDemo    - ``Extensions of Compressed Sensing''
  HDCPNPDDemo  - ``High-Dimensional Centrosymmetric Polytopes with Neighborliness Proportional to
  Dimension''
  MSNVENODemo  - ``Breakdown Point of Model Selection when the Number of Variables Exceeds the
  Number of Observations''
  NPSSULEDemo  - ``Neighborly Polytopes and Sparse Solutions of Underdetermined Linear Equations''
  NRPSHDDemo   - ``Neighborliness of Randomly-Projected Simplices in High Dimensions''
  SNSULELPDemo - ``Sparse Nonnegative Solutions of Underdetermined Linear Equations by Linear
  Programming''
  StOMPDemo    - ``Sparse Solution of Underdetermined Linear Equations by Stagewise Orthogonal
  Matching Pursuit''
\end{verbatim}

These subdirectories have been created following several rules,
which should be followed in making future additions.

\begin{itemize}
\item[1.] Each article gets one subdirectory of \PaperDir.

\item[2.] Each subdirectory contains: (a) the paper itself, (b) a subdirectory housing a
demo.

\item[3.] The files in a subdirectory have stylized names, so that the name indicates the function of the file.

\item[4.] Stylized names are based on a {\it name} and a {\it short prefix}.
The name should be short but descriptive, for example, {\tt Adapt} for scripts associated with the
paper {\it Adapting to Unknown Smoothness via Wavelet Shrinkage}  and the prefix should be a
related tag, just two-characters long, for example {\tt ad}.

\item[5.] The subdirectory name reflects the name you have chosen, for example
{\tt \PaperDir Adapt}.
\end {itemize}

\subsection{Demos}

The Demo subdirectory contains (a) meta-routines that run the whole figure-generating process, (b)
scripts that generate individual figures, (c) datasets the scripts draw on, and (d) specialized
tools, not present in \WaveLab\ proper, for generating the figures.

\subsection{Specialized Tools}

There are several tools available in the {\tt Utilities} directory to help you with writing
scripts.  For example, when creating a display through several {\tt Plot} calls, it is preferable
to use \WaveLab\ functions like {\tt LockAxes} and {\tt UnLockAxes} rather than to use the
low-level \Matlab\ routine {\tt hold}. See Chapter \ref{utilsec} below.

\subsection{Scripting Rules}

\begin {itemize}
\item[I.] One script creates one complete figure, not a series of figures, and
not just a subplot of a figure.

\item[II.]  If several scripts need to work with the same variables -- for
example, if you want a variable to be created in one script and then used in
later scripts -- these variables must be made global (see section 4 below).

\item[III.] No {\tt pause}'s, {\tt print}'s, of {\tt figure} calls in a script.

\item[IV.]  As far as possible try to use the tools in the \WaveLab\ {\tt
Utilities} directory to perform actions like setting axes.
\end {itemize}

Inspection of existing scripts will help in following these rules. If you obey these rules, then
your scripts should be upwardly compatible with script-running engines making more sophisticated
use of the \Matlab\ user interface.

\subsection{Documenting Individual Figures}

Each \dotm\ file for an  individual figure contains a help header
which is displayed in the command window at the time the figure is
generated in the plot window.  This provides a kind of on-line
narrative, or caption.  Here is an example from {\tt ExtCSDemo}:

\begin{verbatim}
% GenFig1 -- ExtCSDemo Figure 1: Error of reconstruction versus
% number of samples for signals with a controlled number of nonzeros.
%
% Data files used: DataL0_20.mat, DataL0_50.mat, DataL0_100.mat
%
% See also: GenDataL0.m
\end{verbatim}

%-----------------------------------------------------------------------------
\section[Adding New Scripts]{Adding New Scripts}
\label{scadd}

To add new demonstration scripts to \PaperDir, having the same
format and effect as {\tt ExtCSDemo}:
\begin {enumerate}
\item Decide on a {\it name} for your demo and a {\it short prefix} for files
implementing your demo. For example, {\tt MyOwnDemo} and {\tt mo}.

\item Create a new subdirectory of \PaperDir. For example, {\tt MyOwn}.

\item Create the following m-files:
\begin{verbatim}
    MyOwnDemo    - starts the Demonstration, invokes Choices
    MyOwnInit    - sets up data structures
    MyOwnFig     - called from Choices
    MyOwnIntro   - help file, explains contents of directories
\end{verbatim}

Suggestion: copy the corresponding files in one of the other subdirectories of
{\tt /Papers} into your new subdirectory, giving them these names; then edit these
files to refer to your own new scripts.

\item Create the scripts which implement your demo: {\tt mofig1.m}, {\tt
mofig2.m}, etc.  The scripts need to follow thee rules mentioned
above in sections 2.2, 2.3 and 2.4.
\end{enumerate}

%-----------------------------------------------------------------------------
\section[Modifying Existing Scripts]{Modifying Existing Scripts}
\label{scmodify}

You might want to modify an existing script for several reasons:
\begin{itemize}
\item To try it out on a different dataset;
\item To try it out with different parameters;
\item To insert a different method in place of the existing method,
using the same dataset.
\end {itemize}

Our rules for script creation should help make this possible.  Some issues to
keep in mind:

First, the script that generates a certain figure might be dependent on
computations done in the process of generating earlier figures.  Therefore, the
script cannot be assumed to work correctly in stand-alone mode. If the script
refers to any global variables then, at a minimum, the corresponding {\tt Init}
script has to be run before the indicated script in order to set global
variables up.

Second, in order to generate a certain effect, it might therefore be necessary
to change earlier scripts, not just the script formally associated with the
figure you are interested in.  The change might have to be in the {\tt Init}
script (to affect global variables), and might possibly have to be in other
scripts as well.

Third, when a set of scripts has been well-written, it should be
necessary {\it only} to change the {\tt Init} script to produce most
changes of the type users will want.

%=============================================
%                Chapter 3
%=============================================
\chapter[Examples]{Examples}
\label{wosec}

Here we describe the contents and architecture of the {\tt
/Examples} subdirectory of \WaveLab. We've included a number of
pedagogical examples in SparseLab, so that the user can familiarize
himself or herself with the software and with our intentions in
providing it. Currently we include:
\begin{verbatim}
      nnfEx            Non-negative Matrix Factorization
      reconstrutionEx  Signal Reconstruction
      RegEx            Model Selection in Regression
      TFDecompEx       Time Frequency Decomposition
\end{verbatim}

Each example is documented on the SparseLab website and can be run
by running the correspondingly named .m file in each directory.

%-------------------------------------------------------------------------------
\section[Examples Philosophy]{Examples Philosophy}
\label{wophilo}

{\tt /Examples} is a subdirectory of /\WLName\ that is much like {\tt Papers} in that it contains a
variety of subdirectories, each of which contains a sequence of scripts generating figures.
However, {\tt Examples} is different in that its primary motivation is {\it not} to reproduce
figures in our own articles.  Instead, its motivation is for more informal, exploratory purposes


%-------------------------------------------------------------------------------
\section[Existing Examples]{Existing Examples}
\label{woexist}

In the current release, version \WLVersion, we distribute the
following Examples:

\begin{verbatim}
 /nnfEx            Non-negative Matrix Factorization

 /reconstrutionEx  Signal Reconstruction

 /RegEx            Model Selection in Regression

 /TFDecompEx       Time Frequency Decomposition
 \end{verbatim}

%-------------------------------------------------------------------------------
\section[Examples Architecture]{Examples Architecture}
\label{woarch}

It is a good idea to follow the same naming practices and file organization
as in the directory \PaperDir.

\subsection{Naming}

In the Regression example,  we use the filenames {\tt RegEx01.m},
{\tt RegEx02.m} after the name of the example directory for the main
script. We try to number figures in an obvious way and to stick with
names no longer than eight characters.

\subsection{Script Contents}

Each file should generate one figure, and should avoid the use of {\tt clg},
{\tt figure}, {\tt print} and {\tt pause}.  This is the same set of rules that
we adhere to in \PaperDir.

\subsection{Meta Routines}

By following the above rules it is easy to write wrapper code to print all
figures or to cycle through all figures.  Such wrapper code typically has
suggestive names like {\tt BBPrintAllFigs} or {\tt BBShowAllFigs}.

%=============================================
%                Chapter 5
%=============================================
\chapter[Datasets]{Datasets}
\label{dasec}

The scripts we have just discussed make use of  several datasets, which are
made available in the directory \DataDir. In this chapter we describe the
architecture of our dataset library.

%---------------------------------------------------------
\section[Dataset Philosophy]{Dataset Philosophy}
\label{daphilo}

We make available datasets through {\it centralized readers}. The idea is that
the knowledge of how to access a dataset should be concentrated in a single
place, and that the access to any dataset should be made in a stereotyped way,
through a simple function call, not through explicit input and output routines.

In this way, if a dataset is available in the system because it has been used
for one script, it automatically becomes available throughout the system for
any other purpose one would wish, without others needing to know the format or
location of the data.

If, in the future, the dataset needs to be moved to some other location in the
file system, or if it needs to be stored in some other format, no scripts that
use the data for demonstrations will need to change.  Instead, one
changes only the code implementing the access method rather than the scripts
which want to use the dataset.

(The alternative is, of course, that any such changes in the future require
rewriting all existing scripts!)

The same philosophy applies for datasets which are synthetic -- those created by
\Matlab\ formulas.  They are accessed in a stereotyped way through access to a
{\it centralized synthesizer}.  In this way, a synthetic signal designed for
one use in one script automatically becomes available for other purposes.

%---------------------------------------------------------
\section[Directory Contents]{Dataset Directory}
\label{daformat}

The \Contents\ file in the {\tt Datasets} directory contains the
following information. It shows that there are several tools for accessing
data, 1-d datasets and 2-d datasets.

It is possible that at some time in the future, we will also have 3-d datasets
(probably movies) or collections of still images.

\begin{verbatim}
%          Data Fabricators
%
%   MakeBlocks        -   Make artificial blocky signal
%   MakeBumps         -   Make artificial bump signal
%   MakeMatrix        -   Make artificial random matrix
%
%
\end{verbatim}

%---------------------------------------------------------
\section[Dataset Documentation]{Dataset Documentation}
\label{dadoc}

Each dataset in the system has a documentation file, with suffix \dotdoc.
Here is an example of a documentation file for a 1-d signal:

\begin{verbatim}
caruso.asc -- Digital signal of Caruso singing

Access
    Enrico = ReadSignal('Caruso');

Size
    50,000 by 1

Sampling Rate
    8192 Hz

Description
    In MATLAB, the command sound(Enrico,8192) will play this sound
    back at the right pitch.

Source
    Obtained by anonymous FTP from the xwplw package
    developed by R.R. Coifman and Fazal Majid at Yale University.
    You can get this X-windows adapted waveform analysis
    package by anonymous FTP to math.yale.edu.
\end{verbatim}

Here is an example of a documentation file for a 2-d image:
\begin{verbatim}
canaletto.raw -- Gray-scale image of Canaletto painting

Access
    Canal = ReadImage('Canaletto');

Size
    512 by 512

Gray Levels
    256

Description
    This image was used in an article by P. Perona and J. Malik,
    "Scale-Space Filtering by Anisotropic Diffusions," IEE PAMI.

Source
    Obtained from John Canny and Jitendra Malik, of EECS at
    U.C. Berkeley.
\end{verbatim}

\pagebreak
You will notice the following fields in the documentation:
\begin{enumerate}
\item {\it Title}. A one-line header at the start of the file, giving the
filename, and, after two hyphens, descriptive text.

\item {\it Access}. A code fragment indicating the stereotyped access method.

\item {\it Size}. The size of the signal or image.

\item {\it Gray Levels}. Applicable for Images only.

\item {\it Sampling Rate}. Applicable for Sounds only.

\item {\it Source}. Indication of the original source of the data.

\item {\it Description}. Additional description of the data.
\end{enumerate}

%---------------------------------------------------------
\section[Adding New Datasets]{Adding New Datasets}
\label{daadd}

To add new datasets to \WaveLab, do the following:

\begin{itemize}
\item[1.] {\it Installation.}  Place the dataset, in stereotyped format, in the
{\tt Datasets} directory.  Modify one of the existing access functions to read
in the dataset. (You can, in a pinch, place the dataset elsewhere, or keep it
in a nonstandard format).

\item[2.] {\it Documentation.}  Insert a \dotdoc\ file in the {\tt Datasets}
directory to explain the dataset.
\end {itemize}

To add a new synthetic matrix type to \WaveLab, simply modify the
function {\tt MakeMatrix}, by inserting a new case in the ``compound
if''; the new case tests for a new, previously unused name, and
contains a formula defining the signal in that case. Add a separate
function, similar to {\tt UniformSphericalMatrix} for example, with
the build instructions. It is best if the formula is designed to
work the same way the other formulas work -- to produce an output at
any given signal length or image extent.

%-----------------------------------------------------------------
\section[Dataset Sources]{Dataset Sources}
\label{dasource}

We would like to take this opportunity to thank the sources of our datasets.
We reprint here from the file {\tt THANKS.m} in \DocDir.
\begin{verbatim}
%  Contributors of Data
%       Yaakov Tsaig
\end{verbatim}

%=============================================
%                Chapter 6
%=============================================
\chapter[Documentation]{Documentation}
\label{docsec}

There has been extensive concern for the documentation of the code in \WaveLab.  We try to use all
the features of \Matlab\ as well as other features to produce a coherent, understandable system.

%---------------------------------------------------------
\section[Help Headers]{Help Headers}
\label{dochelp}

Each function in the \WaveLab\ system has documentation contained
inside the \dotm\ file with its \Matlab\ code. This documentation
can be accessed on-line by typing {\tt help Name} where {\tt Name}
is the name of the function. For example, typing {\tt help SolveMP}
gives:

\begin{verbatim}
% SolveMP: Matching Pursuit (non-orthogonal)
% Usage
%   [sol iters activationHist] = SolveMP(A, b, maxIters, NoiseLevel, verbose)
% Input
%   A           dictionary (dxn matrix), rank(A) = min(d,n) by assumption
%   y           data vector, length d.
%   maxIters    number of atoms in the decomposition
%   NoiseLevel  estimated norm of noise, default noiseless, i.e. 1e-5
%   verbose     1 to print out detailed progress at each iteration, 0 for
%               no output (default)
% Outputs
%    sol             solution of MP
%    iters           number of iterations performed
%    activationHist  Array of indices showing elements entering
%                    the solution set
% Description
%   SolveMP implements the greedy pursuit algorithm to estimate the
%   solution of the sparse approximation problem
%      min ||x||_0 s.t. A*x = y
% See Also
%   SolveOMP
% References
%   Matching Pursuit With Time-Frequency Dictionaries (1993) Mallat, Zhang
%   IEEE Transactions on Signal Processing
%\end{verbatim}

This illustrates the main components of the format we have adopted: a one-line
{\it help header}, and sections for {\it Usage}, {\it Inputs}, {\it Outputs},
{\it Side Effects}, {\it Description}, {\it Examples}, {\it Algorithm}, {\it
See Also} and {\it References}.

\begin{itemize}
\item[1.] {\it Header}. The first line of the help header is called the {\tt
H1} line by the \Matlab\ folks. It is special to \Matlab, and to \WaveLab.
When you use the {\tt lookfor} command,  \Matlab\ examines this line for each
\dotm\ file in its path to find text matching the request. When a release of
\WaveLab\ is built, these lines are compiled and sorted in alphabetical order to
make files in the documentation directory. Format: a percent sign, a single
blank, the name of the function, a blank followed by double hyphens and a
blank, and a short description of the function.  The description should contain
as many helpful keywords as possible.

\item[2.] {\it Usage}. Here, indicate the calling prototype.  Format: the
output argument(s) (enclosed within square brackets if there is more than one
output argument), an equals sign, the function name followed by the input
argument(s) enclosed within parentheses.  Optional input arguments are enclosed
within square brackets.

\item[3.] {\it Inputs}.  Here, one line per input variable, indicating the name
of the variable, the formal data type and the interpretation. Also, indicate if
the input is optional by enclosing it within square brackets.

\item[4.] {\it Outputs}.  Here, one line per output variable, indicating the
name of the variable, the formal data type and the interpretation.

\item[5.] {\it Side Effects}.  Here, indicate any side effects the routine may
have (graphics, sound, etc.).  Omit if the function has no side effects.

\item[6.] {\it Description}. Here, describe what the function does in as much
detail as possible.

\item[7.] {\it Examples}.  Here, list examples of how the function is called in
practice.  This field is optional.

\item[8.] {\it Algorithm}.  Here, describe the algorithm used by the function.
This field is optional.

\item[9.] {\it See Also}. Here, mention other routines which this routine calls
or which call this one, or routines with a special relationship to this
function.  This field is optional.

\item[10.] {\it References}.  Here, list references from which the user may
obtain further information about the function.  This field is optional.
\end {itemize}

%---------------------------------------------------------
\section[Documentation Directory]{Documentation Directory}
\label{docdir}

The directory \DocDir\ contains a variety of information about \WaveLab. There
are a number of general files, which describe various terms and conditions and
goals.  The contents of any of these files may be examined by typing its name.

\begin{verbatim}
% ADDINGNEWFEATURES       -  How to Add New Features to SparseLab
% BUGREPORT               -  How to report bugs about SparseLab
% COPYING                 -  SparseLab Copying Permissions
% DATASTRUCTURES          -  Basic data structures in SparseLab
% FEEDBACK                -  Give feedback about SparseLab
% GETTINGSTARTED          -  Ideas for getting started with SparseLab
% INSTALLATION            -  Installation of SparseLab
% LIMITATIONS             -  SparseLab known limitations
% PAYMENT                 -  No Charge for SparseLab Software
% REGISTRATION            -  SparseLab Registration
% SUPPORT                 -  SparseLab Support
% THANKS                  -  Thanks to contributors
% VERSION                 -  Part of SparseLab Version v$VERSION$
% WARRANTY                -  No Warranty on SparseLab software
\end{verbatim}


To add or modify the first group of files, very little is required.  Simply add
new files.  The second group of files, being automatically generated at build
time, should not ordinarily be modified.  Instead, modify the source from which
they are automatically compiled.

Because of the automatic build process, it is important to maintain
the integrity of certain files.  These include: \begin{itemize}
\item Contents files.  Every directory should have a \Contents\ file.  When
adding a new function to a directory, be sure to add it to the directory's
Contents file as well.
\item H1 Lines of Help documents.  Every \dotm\ file should contain a help
header, and the {\tt H1} line of the help header should follow the rules
specified above.
\item \$VERSION\$ marker.  Every \Contents\ file has, in the {\tt H1}
line, a description of what the directory contains, as well as a version marker.
The text \$VERSION\$ is replaced, automatically upon build, by the current
version number.
\end{itemize}


%---------------------------------------------------------
\section[Examples Directory]{Examples Directory}
\label{workdir}

Another useful component of the system documentation is the {\tt
/Examples} directory, which contains scripts that exercise the
software in various ways.

The user can look through the graphics generated by this
documentation and, upon seeing something interesting, inspect the
corresponding script to see how the graphic was created.

Currently, the {\tt /Workouts} directory contains three
subdirectories:

\begin{verbatim}
      nnfEx            Non-negative Matrix Factorization
      reconstrutionEx  Signal Reconstruction
      RegEx            Model Selection in Regression
      TFDecompEx       Time Frequency Decomposition
\end{verbatim}

%---------------------------------------------------------
\section[\TeX\ Documents]{\TeX\ Documents}
\label{doctex}

The system also comes with several documents, written in \TeX, which
function as manuals for system-maintenance people.


The file {\tt SparseMacros.tex} within {\tt SparseLab Documentation}
contains macros that define the current version of \WaveLab,
filenames, file sizes, file locations, etc. This file should be
modified appropriately for new releases of \WaveLab.  It is included
by all the documents described below.

\subsection{About SparseLab}

{\it About SparseLab} helps a new user with installing and getting
started with \WaveLab.  The corresponding pdf and postscript
documents are available at:\\ {\tt \AboutWLAddress}. The source is
written in \LaTeX.  It is contained within the {\tt About SparseLab}
folder {\tt Documentation}.


%---------------------------------------------------------
\subsection{Architecture}

You are currently reading the {\it SparseLab Architecture} document.
It contains system-level information about the \WaveLab\
distribution. The corresponding postscript document is available via
{\tt \WLArchitectureAddress}.  The source is written in \LaTeX.  It
is contained within the {\tt SparseLab Architecture} folder in {\tt
SparseLab Documentation}.

%=============================================
%                Chapter 8
%=============================================
\chapter[Utilities]{Utilities}
\label{utilsec}

Several utilities are available in \WaveLab\ mainly for the purpose of
centralizing various programming idioms.  If \WaveLab\ is ever to be ported to
Octave, for example, these allow one to modify only the utilities to the new
platform and achieve the desired effect of platform-independent scripts.

The current \Contents\ file for \UtilDir\ goes as follows:
\begin{verbatim}
%   Contents.m            -   This file
%   aconv.m              -   Convolution Tool for Two-Scale Transform
%   AutoImage.m          -   Automatic Scaling for Image Display
%   DownDyadHi.m         -   Hi-Pass Downsampling operator (periodized)
%   DownDyadLo.m         -   Lo-Pass Downsampling operator (periodized)
%   dyad.m               -   Index entire j-th dyad of 1-d wavelet xform
%   dyadlength.m         -   Find length and dyadic length of array
%   FWT_PO.m             -   Forward Wavelet Transform (periodized, orthogonal)
%   FWT_TI.m             -   translation invariant forward wavelet transform
%   iconv.m              -   Convolution Tool for Two-Scale Transform
%   IWT_PO.m             -   Inverse Wavelet Transform (periodized, orthogonal)
%   IWT_TI.m             -   translation invariant forward wavelet transform
%   LockAxes.m           -   Version-independent axis command
%   lshift.m             -   Circular left shift of 1-d signal
%   MakeONFilter.m       -   Generate Orthonormal QMF Filter for Wavelet Transform
%   MirrorFilt.m         -   Apply (-1)^t modulation
%   Noisemaker.m         -   Add Noise to Signal
%   NormNoise.m          -   Estimates noise level, Normalize signal to noise level 1
%   packet.m             -   Packet table indexing
%   PlotSpikes.m         -   Plot 1-d signal as baseline with series of spikes
%   PlotWaveCoeff.m      -   Spike-plot display of wavelet coefficients
%   RegisterPlot.m       -   Add legend with file name, date, flag
%   reverse.m            -   Reverse order of elements in 1-d signal
%   rshift.m             -   Circular right shift of 1-d signal
%   ShapeAsRow.m         -   Reshape 1d vector as row
%   ShapeLike.m          -   Reshape first argument like second argument
%   TIDenoise.m          -   Translation invariant denoising of a 1-D signal
%   twonorm.m            -   Computes ||v||_2
%   UnlockAxes.m         -   Version-independent axis command
%   UpDyadHi.m           -   Hi-Pass Upsampling operator; periodized
%   UpDyadLo.m           -   Lo-Pass Upsampling operator; periodized
%   UpSample.m           -   Upsampling operator
\end{verbatim}

The functions of these utilities can be loosely classified into the categories: {\it Graphics},
{\it Random Numbers}, {\it Shaping Arrays}, and {\it Scripting}.


%=============================================
%                Chapter 9
%=============================================
\chapter[Source and Build]{Source and Build}
\label{srcbld}

This chapter describes how \WaveLab\ source is compiled into archives for
distribution.

%---------------------------------------------------------
\section[Development System]{Development System}
\label{devsys}

\begin{itemize}
\item[1]The source for \WaveLab\ development has several components in
different directories:

\item[2] {\it TeX Source} in a directory named {\tt Documentation} inside the
{\tt SparseLab100\} folder.

\item[3] {\it Shell Source} in a directory named {\tt shell\_tools} inside the
{\tt SparseLab100\} folder.
\end{itemize}

\section[Shell Tools]{Shell Tools}
\label{shelltools}

Here is an up-to-date list of the high-level files:

\begin{verbatim}
 append_footer.sh        -   Appends a footer to all non-Contents .m files
 SparseLab_Footer.txt    -   the footer that gets appended
\end{verbatim}


These can be used outside of the Master build process.


%---------------------------------------------------------
\section[Standard Release]{Standard Release}
\label{stdrel}

The process of building a ``standard'' release involves:

\begin{itemize}
\item[1.] Appending copyright notices and date-of-modification information to
all files in the library;
\item[2.] Adding the Matlab Source files to a zip file. This .zip
file is the final version that will be made available on the WWW
sites.
\end{itemize}

%---------------------------------------------------------
\section[Compiling \dotps]{Compiling \dotps}
\label{compps}

The {\tt Documentation} directory within {\tt SparseLab\_Master}
contains one folder for each of the \WaveLab\ documents: {\it About
SparseLab} and {\it SparseLab Architecture}. These folders contain
the \LaTeX\ code for the documents, which are compiled into \dotps\
and .pdf files.  These \dotps\ and .pdf files are then made
available on the WWW sites.


%---------------------------------------------------------
\section[Distribution]{Distribution}
\label{distribution}

The uniform download process is chosen for Sparselab. By uniform download process we mean that all
the users, independent of the platforms they are using, download the file SparseLabvers.zip in
which vers is replaced by the version of the Sparselab. Then during the installation process
Sparselab will recognize their platform. Therefore, there is no need for releasing different files
for different platforms anymore. The only thing that the distributor should do is just putting the
SparseLabvers.zip on the WWW site.

Because of the size of some of the precomputed data files included in SparseLab, two separate
download packages, SparseLabvers\_DataSupplementExtCS.zip and
Sparselabvers\_DataSupplementStOMP.zip (where vers is replaces by the appropriate version number)
are created for download. Optimally, all three .zip files should be downloaded and installed
together.


%=============================================
%                Chapter 10
%=============================================
\chapter[Distribution and Maintenance]{Distribution and Maintenance}
\label{dist}

This chapter describes how \WaveLab\ is distributed and maintained.

%---------------------------------------------------------
\section[Archive Directory]{Archive Directory}
\label{devarch}

The {\tt Archive} directory within {\tt SparseLab} is a depository for old versions of the software
and documentation.

%---------------------------------------------------------
\section[SparseLab Account]{SparseLab Account}
\label{slacct}

An account named {\tt SparseLab} is maintained on the leland system at stanford.edu, as is the
website. The account serves several varied purposes:

\begin{enumerate}

\item[1.] The sub-directory {\tt WWW} holds the files used to maintain
our Web page.

\item[3.] The current version of \WaveLab\ is always present on this account in
the sub-directory {\tt SparseLab}.

\item[4.] Feedback -- questions, comments, suggestions, etc. -- may be sent to
the development team by e-mailing \eWaveLab.
\end{enumerate}


%---------------------------------------------------------
\section[Web Page]{Web Page}
\label{webpage}

The URL of the \WaveLab\ WWW page is \WLWEB.  The html files for the home page are stored in the
Documentation subdirectory of SparseLab100/.
 The home page is constantly changing and evolving.
New versions and updates are always announced on the home page.

\end{document}
