% AboutSparseLab.tex

\documentclass{article}

\addtolength{\textwidth}{1in}
\addtolength{\oddsidemargin}{-0.5in}
\addtolength{\textheight}{1.6in}
\addtolength{\topmargin}{-0.8in}
\usepackage{epsf}
\usepackage{graphicx}

\begin{document}

\input SparseMacros.tex

\title{About SparseLab}
\author{
\bigskip \\
David Donoho, Victoria Stodden, Yaakov Tsaig\\
Stanford University}
\date{Version 2.1\\
August 24, 2007}

\maketitle

\begin{abstract}
{\bf Changes and Enhancements for Release 2.1 (August 2007)}: 2 papers have been added to Sparselab 2.1: "Bayesian 
Compressive 
Sensing" and "Multi-Task Compressive Sensing".

In the Solvers/ subdirectory a new version of SolveBP.m that updated it's compatibility with the ExtCS demo, and the 
files SolveFastCS.m and FastCSMatrix.m have been added.


{\bf Changes and Enhancements for Release 2.0 (March 2007)}: 4 papers have been added to Sparselab 200: "Fast
Solution of l1-norm Minimization Problems When the Solutions May be Sparse"; "Why Simple Shrinkage
is Still Relevant For Redundant Representations"; "Stable Recovery of Sparse Overcomplete
Representations in the Presence of Noise"; "On the Stability of Basis Pursuit in the Presence of
Noise."

\WaveLab\ is a library of \Matlab\ routines for finding sparse solutions to underdetermined
systems. The library is available free of charge over the Internet. Versions are provided for
Macintosh, UNIX and Windows machines. Downloading and installation instructions are given here.

\WaveLab\ has over 400 \dotm\ files which are documented, indexed and cross-referenced in various
ways.  In this document we suggest several ways to get started using \WaveLab: (a) trying out the
pedagogical examples, (b) running the demonstrations, which illustrate the use of SparseLab in
published papers, and (c) browsing the extensive collection of source files, which are
self-documenting.

\WaveLab\ makes available, in one package, all the code to reproduce all the figures in the
included published articles. The interested reader can inspect the source code to see exactly what
algorithms were used, and how parameters were set in producing our figures, and can then modify the
source to produce variations on our results. \WaveLab\ has been developed, in part, because of
exhortations by Jon Claerbout of Stanford that computational scientists should engage in ``really
reproducible'' research.

This document helps with installation and getting started, as well as describing the philosophy,
limitations and rules of the road for this software.
\end{abstract}

{\bf Acknowledgment of Support.} This work was partially supported
by NSF DMS-05-05303 (Stanford).

\pagebreak

\tableofcontents

\pagebreak

\section{Introduction}

\WaveLab\ is a library of \Matlab\ routines for finding sparse solutions to underdetermined
systems. The library provides the research community with open source tools for sparse
representation, as well as being the basis for research by the authors, and may be used to
reproduce the figures in their published articles, and to redo those figures with variations in the
parameters.

The library is available free of charge over the Internet by WWW access; instructions are given
below.  The material is, however, copyrighted, so that advance permission is required for any
commercial use.

The package approaches the problem of sparse representation from both signal processing and
statistical viewpoints. The user is free to choose the terminology he or she is comfortable with.
SparseLab incorporates software for several published solvers, for example Michael Saunders'
Primal-Dual method for Optimization with Convex Objectives, Mallat and Zhang's Matching Pursuit,
Donoho and Johnstone's Iterative Hard and Soft Thresholding, Efron et al's Least Angle Regression,
and a number of others.

In addition to routines finding sparse solutions to systems, the library contains scripts which
give a quick examples in a variety of different settings. We believe that by studying these scripts
one can quickly learn the practical aspects of sparse representation and one can learn how to use
the \WaveLab\ software library

In this guide we give information which will help you access and install the software on your
machine and get started in exploring the resources contained in the \WaveLab\ distribution. We also
explain the philosophy which underlies our distribution of the software, and some of the fine print
associated with the software.

There are other resources for obtaining information about \WaveLab. First, there is a {\it
SparseLab Architecture} guide which gives details about how \WaveLab\ is constructed and
maintained. Secondly, we give more information on the SparseLab website.

This body of software is under continuing development by a team of researchers supported by a grant
from the NSF, and from other sponsors. We conduct our research with the idea, from the beginning,
that we will implement our tools in \WaveLab. We believe that the discipline this entails makes our
research of a higher quality than otherwise possible.

We welcome your suggestions for further enhancements, and any contributions you might make.

\pagebreak

\section{Access and Installation}

The \WaveLab\ library contains \dotm\ files (\Matlab\ code), datasets, documentation scripts and
workouts (both also \dotm\ files) for reproducing the figures in articles by the authors.

The whole library consists of over 400 files. It requires more than \MinSize\ and less than
\MaxSize\ space on disk once it is downloaded, decompressed and installed. The largest data files
for two demos are included in separate packages: Sparselab100\_DataSupplementExtCS.zip and
Sparselab100\_DataSupplementStOMP.zip -- the majority of the size comes from these components.

This documentation refers to Version \WLVersion\ of \WaveLab.

\subsection{Platform-Specific Information}

\WaveLab\ is available for use in \Matlab\ 6.x or 7.x on three
different platforms: Windows XP or 2000, UNIX/Linux and Macintosh.
The package is made available as a compressed archive, in a \dotzip\
format.\\
 You do have to know about one convention used in the
documentation. We always use the UNIX pathname conventions rather
than PC or Macintosh, e.g.\ {\tt Matlab/Toolbox/\WLDir} rather than
{\tt Matlab\BS Toolbox\BS SparseLab} or {\tt
Matlab:Toolbox:WaveLab}. You have to transliterate what we say into
the version appropriate for your platform.

\subsection {WEB Acess}
To download the compressed archive from the web, point your web
browser to \WLWEB\ to access the \WaveLab\ web-page. Once there,
mouse click the "Download" link in the left frame.

\subsection{Installation}

In this section we first describe the installation process in narrative form,
and later give a step-by-step checklist.

Once the appropriate compressed archive has been transferred to your
machine, it should be decompressed and installed. You will need an
appropriate software to decompress \dotzip\ file \WLName \dotzip. On
a personal computer (Macintosh or Windows), the archives should be
decompressed and installed as a subdirectory of the {\tt Toolbox}
directory inside the {\tt matlab} folder. On a UNIX workstation or
server, the archives could either be installed in the systemwide
{\tt matlab} directory, if you have permission to do this, or in
your own personal {\tt matlab} directory, if you do not.

Once the actual files are installed, you should have a number of files and
subdirectories in the directory \WLDir. If you look in the files
\Contents\ inside of the \WLDir\ directory, you will see a plan of what is inside:

\begin{verbatim}
% SparseLab Main Directory, Version 100
%
% This is the main directory of the SparseLab package.
%
%              .m files in this directory
%
%   Contents.m           -    This file
%   SparsePath.m         -    Sets up global variables and pathnames
%
%              Subdirectories
%   Documentation        -    System-Wide Documentation
%     /About SparseLab
%     /SparseLab Architecture
%   Examples             -    Detailed examples of SparseLab finding sparse solutions
%     /nnfEX             -    Nonnegative Factorization Example
%     /reconstructionEx  -    Signal Reconstruction Example
%     /RegEx             -    Regression Example
%     /TFDecompEx        -    Time-Frequency Reconstruction Example
%   Papers               -    Scripts for reproducing figures in published articles
%     /ExtCS             -    figures for "Extensions of Compressed Sensing"
%     /HDCPNPD           -    figures for "High-Dimensional Centrosymmetric
%                             Polytopes with Neighborliness Proportional to Dimension"
%     /NPSSULE           -    table in "Neighborly Polytopes and Sparse
%                             Solutions of Underdetermined Linear Equations"
%     /NRPSHD            -    figures for "Neighborliness of
%                             Randomly-Projected Simplices in High Dimensions"
%     /SNSULELP          -    figures for "Sparse Nonnegative Solutions of
%                             Underdetermined Linear Equations by Linear Programming"
%   Solvers              -    Sparse solver packages
%   Tests                -    Simple pedagogical workouts
%   Utilities            -    General tools for developers and users
%   shell_tools          -    Tools for use during the SparseLab build process

%
% Part of SparseLab Version:100
% Created Tuesday March 28, 2006
% This is Copyrighted Material
% For Copying permissions see COPYING.m
% Comments? e-mail sparselab@stanford.edu
\end{verbatim}

Make a local directory listing to see if your hard disk actually has
these files and subdirectories.

\subsection{Pathnames}

\Matlab\ can automatically, at startup time, make all the \WaveLab\ software
available. The script \WavePath\ is provided as part of \WaveLab\ to enable this
feature. It should be invoked from the user's \Startup\ file.

\begin{itemize}
\item[PC] \Startup\ is located in the {\tt matlab\BS local} directory on MS-Windows.
Insert the line {\tt SparsePath} in that file, and put a copy of
\WavePath\ in that directory.

\item[Mac] \Startup\ may be located anywhere inside the {\tt Matlab} directory
on Macintosh. Insert the line {\tt SparsePath} in that file. Since
\WaveLab\ contains a \Startup\ file, if you have no other \Startup\
file, there is nothing to do once \WaveLab\ is installed.

\item[Unix] This file is located in the  {\tt matlab} subdirectory of your home
directory on UNIX. If you don't have such a subdirectory, use {\tt
mkdir $\sim$/matlab} to make one. Create a file named \Startup\ and
insert the line {\tt SparsePath} in that file.  Then put a copy of
\WavePath\ in that directory.

\end{itemize}

\subsection{Checklists}

To reinforce the above points, we furnish here step-by-step installation
checklists.

\subsubsection{UNIX Checklist}
\begin{enumerate}
\item[1.] Binary Download the archive to the directory you want
\WaveLab\ to reside.
\item[2.] Uncompress the archive: {\tt SparseLab100.zip}
\item[3.] Decide where you want the \WaveLab\ directory to reside. It will have
a number of subdirectories and occupy at least \MinSize\ disk space.
\item[4.] After you decompress the file for your machine, you should
have the following directory structure.\\
\WLName \\
\WLName / Solvers\\
\WLName / Datasets \\
\WLName / Documentation\\
\WLName / Papers\\
\WLName / Examples\\
\WLName / Utilities \\
\item[5.] Copy all the SparseLab files from the place you put the original SparseLab
 archive (for example /tmp) to their final destination, for example in your home
 directory
    \~ user/matlab/\WLName.

\item[6.] Launch Matlab; In Matlab set the current path to
matlabroot/toolbox/\WLName\ or alternatively copy the file
\WavePath\ from $<$ MatlabToolboxPath $>$ /\WLName\ to
$<$MatlabToolboxPath$>$ /local

\item[7.] Run \WavePath; If the default pathname is not right the
program will ask you to enter the correct path.
\item[8.] Type {\tt installMEX} to compile and install the \dotmex\ files.
\end{enumerate}

{\it Trouble-Shooting UNIX:} Compare the output of {\tt ls -r
SparseLab100} with {\tt Documentation}  to see if you have all the
files. Compare the output of the \Matlab\ command {\tt path} with
the list above to see if you have all the directories in your path.

\subsubsection{Macintosh Checklist}
To follow these instructions you will need:
\begin{itemize}
\item[(1)] A Macintosh running MacOS 10.3 or later.
\item[(2)] A program which can unzip \dotzip\b
file.
\item[(3)] Matlab 6.x or 7.x for Mac.
\end{itemize}
Steps:\\
\begin{itemize}
\item[1.] Binary Download the file {\tt \WLName \dotzip} to your
Macintosh.
\item[2.] Extract the archive to the Toolbox folder of your Matlab
folder. After you extract the file you should have the following
subdirectory structure: \\
\WLName \\
\WLName / Solvers\\
\WLName / Datasets \\
\WLName / Documentation\\
\WLName / Papers\\
\WLName / Examples\\
\WLName / Utilities \\

\item[3.] Launch Matlab; In Matlab set the current path to
matlabroot/toolbox/\WLName\ or alternatively copy the file
\WavePath\ from $<$ MatlabToolboxPath $>$/\WLName\ to
$<$MatlabToolboxPath$>$/local
\item[4.] Run \WavePath\ at the command prompt to start \WaveLab.
You will see a "Welcome to SparseLab" message as shown in the section Success below.
\end{itemize}
Note: \\
\begin{enumerate}
\item If you want to automatically load \WLName\ upon the start-up
copy the file \WavePath\ from the folder \WLName\ to the folder
Matlab/Toolbox/local. Determine if you have any file named startup.m
besides the one that is in \WLName\ directory. If you don't, go to
step 3.\\
 \item if you have \Startup\ , then copy the contents of \WavePath\
 into this file.
 \item If you don't have any \Startup\ , then copy the file \Startup\ from
 \WLName\ directory to
 $<$MatlabToolboxPath$>$/local
 \end{enumerate}

\subsubsection{PC Checklist}
To follow these instructions you will need:
\begin{itemize}
\item[(1)] An Intel Platform running Windows 2000 or XP.
\item[(2)] A program such as Winzip which can unzip \dotzip\
file.
\item[(3)] Matlab 6.x or 7.x for Windows.
\end{itemize}
\begin{enumerate}
\item[1.] Binary Download the file {\tt \WLName \dotzip} to your
PC.
\item[2.] Extract the archive to the Toolbox folder of your Matlab
folder. After you extract the file you should have the following
subdirectory structure: \\
\WLName \\
\WLName / Solvers\\
\WLName / Datasets \\
\WLName / Documentation\\
\WLName / Papers\\
\WLName / Examples\\
\WLName / Utilities \\

\item[3.] Launch Matlab; In Matlab set the current path to
matlabroot\BS toolbox \BS \WLName\ or alternatively copy the file
\WavePath\ from $<$ MatlabToolboxPath $>$ \BS \WLName\ to
$<$MatlabToolboxPath$>$ \BS local
\item[4.] Run \WavePath\ at the command prompt to start \WaveLab. You will see a
"Welcome to SparseLab" message as shown in the section Success
below.
\end{enumerate}

Note: \\
\begin{enumerate}
\item If you want to automatically load \WLName\ upon the start-up
copy the file \WavePath\ from the folder \WLName\ to the folder
Matlab \BS Toolbox \BS local. Determine if you have any file named
startup.m besides the one that is in \WLName\ directory. If you
don't go to
step 3.\\
 \item if you have \Startup\, then copy the contents of \WavePath\
 into this file.
 \item If you don't have any \Startup\, then copy the file \Startup\ from \WLName\ directory to
 $<$MatlabToolboxPath$>$ \BS local
 \end{enumerate}


\subsection{Success}

When you have a successful installation,
you should see something like the following when you invoke \Matlab:

\begin{verbatim}
Welcome to SparseLab v 100

Setting Global Variables:
   global MATLABVERSION = 7
   global SparseLABVERSION = 100
   global SparseLABPATH = C:\Program Files\MATLAB704\work\SparseLab\SparseLab100\
   global PATHNAMESEPARATOR = "\"
   global PREFERIMAGEGRAPHICS = 1

SparseLab 100 Setup Complete

Currently available browsers for reproducing figures from the
following papers:
   ExtCSDemo    - demo for paper "Extensions of Compressed Sensing"
   HDCPNPDDemo  - demo for paper "High-Dimensional Centrosymmetric Polytopes with
                  Neighborliness Proportional to Dimension"
   MSNVENODemo  - demo for paper "Breakdown Point of Model Selection When the
                  Number of Variables Exceeds the Number of Observations"
   NPSSULEDemo  - demo for paper "Neighborly Polytopes and Sparse Solutions of
                  Underdetermined Linear Equations"
   NRPSHDDemo   - demo for paper "Neighborliness of Randomly-Projected Simplices
                  in High Dimensions"
   SNSULELPDemo - demo for paper "Sparse Nonnegative Solutions of Underdetermined
                  Linear Equations by Linear Programming"
   StOMPDemo    - demo for paper "Sparse Solution of Underdetermined Linear Equations
                  by Stagewise Orthogonal Matching Pursuit"

Currently available examples:
   Nonnegative Factorization
   Signal Reconstruction
   Regression Example
   Time-Frequency Separation

For more information, please visit:
   http://sparselab.stanford.edu

Please ignore the following message if WaveLab has been installed.

There are SparseLab functions which call WaveLab functions. We
recommend that the users download WaveLab from the website
   http://www-stat.stanford.edu/~wavelab
and install the package in the directory
   C:\Program Files\MATLAB704\toolbox


\end{verbatim}

\section{Getting Started}

There are several ways to get started with \WaveLab. First, you can
snoop around the directory structure to see what's there.  Second,
you can try running some of the demos to see what they do.  Third,
you can try the pedagogical examples.

\subsection{Snooping}

If you just snoop around in the \WaveLab\ file structure, you will notice many
directories and a great range of different information about the system itself
and what it can do.  We list here some basic facts.

\subsubsection{Contents Files}

Each directory has a {\tt Contents.m} file, which explains the
contents and purpose of that directory. The directory {\tt Solvers}
contains the central program solver tools; its {\tt Contents.m} file
looks as follows:
\begin{verbatim}
% SparseLab Solvers Directory
%
% This is directory houses the solvers for the SparseLab package.
%
%              .m files in this directory
%
%   Contents.m           -    This file
%   fdrthresh.m          -    Uses the False Discovery Rate to Threshold a
%                             Signal
%   HardThresh.m         -    Implements Hard Thresholding
%   lsqrms.m             -    Iterative least squares
%   pdco.m               -    Primal-Dual Barrier Method for Convex
%                             Objectives (Michael Saunders 2003)
%   pdcoSet.m            -    creates or alters options structure for
%                             pdco.m
%   SoftThresh.m         -    Soft Thresholding
%   SolveBP.m            -    Basis Pursuit
%   SolveIRWLS.m         -    Iteratively ReWeighted Least Squares
%   SolveIST.m           -    Iterative Soft Thresholding
%   SolveISTBlock.m      -    Iterative Soft Thresholding, block variant
%                             with least squares projection
%   SolveLasso.m         -    Implements LARS/Lasso Agorithms
%   SolveMP.m            -    Matching Pursuit
%   SolveOMP.m           -    Orthogonal Matching Pursuit
%   SolveStepwise.m      -    Forward Stepwise
%   SolveStepwiseFDR.m   -    Forward Stepwise with FDR Threshold
%   SolveStOMP.m         -    Stagewise Orthogonal Matching Pursuit
%

%
% Part of SparseLab Version:100
% Created Tuesday March 28, 2006
% This is Copyrighted Material
% For Copying permissions see COPYING.m
% Comments? e-mail sparselab@stanford.edu
%
\end{verbatim}

\subsubsection{Help for Functions}

Each function in \WaveLab\ has help documentation.  For example,
{\tt SolveMP} is Mallat and Zhang's Matching Pursuit algorithm. If
you are in \Matlab\ and type {\tt help SolveMP}, \Matlab\ will type
out the following documentation:

\begin{verbatim}
% SolveMP: Matching Pursuit (non-orthogonal)
% Usage
%   [sol iters activationHist] = SolveMP(A, b, maxIters, NoiseLevel, verbose)
% Input
%   A           dictionary (dxn matrix), rank(A) = min(d,n) by assumption
%   y           data vector, length d.
%   maxIters    number of atoms in the decomposition
%   NoiseLevel  estimated norm of noise, default noiseless, i.e. 1e-5
%   verbose     1 to print out detailed progress at each iteration, 0 for
%               no output (default)
% Outputs
%    sol             solution of MP
%    iters           number of iterations performed
%    activationHist  Array of indices showing elements entering
%                    the solution set
% Description
%   SolveMP implements the greedy pursuit algorithm to estimate the
%   solution of the sparse approximation problem
%      min ||x||_0 s.t. A*x = y
% See Also
%   SolveOMP
% References
%   Matching Pursuit With Time-Frequency Dictionaries (1993) Mallat, Zhang
%   IEEE Transactions on Signal Processing
%
\end{verbatim}

\subsubsection{Source Browsing}

All the algorithms in \WaveLab\ are available for inspection. For
example, if you are in \Matlab\ and type {\tt type SolveMP} you get
the following documentation:

\begin{verbatim}
function [sol iters activationHist] = SolveMP(A, y, maxIters,
NoiseLevel, verbose)
% SolveMP: Matching Pursuit (non-orthogonal)
% Usage
%   [sol iters activationHist] = SolveMP(A, b, maxIters, NoiseLevel, verbose)
% Input
%   A           dictionary (dxn matrix), rank(A) = min(d,n) by assumption
%   y           data vector, length d.
%   maxIters    number of atoms in the decomposition
%   NoiseLevel  estimated norm of noise, default noiseless, i.e. 1e-5
%   verbose     1 to print out detailed progress at each iteration, 0 for
%               no output (default)
% Outputs
%    sol             solution of MP
%    iters           number of iterations performed
%    activationHist  Array of indices showing elements entering
%                    the solution set
% Description
%   SolveMP implements the greedy pursuit algorithm to estimate the
%   solution of the sparse approximation problem
%      min ||x||_0 s.t. A*x = y
% See Also
%   SolveOMP
% References
%   Matching Pursuit With Time-Frequency Dictionaries (1993) Mallat, Zhang
%   IEEE Transactions on Signal Processing
%


if nargin < 5,
    verbose = 0;
end

if nargin < 4,
    NoiseLevel = 1e-5;
end

if nargin < 3,
    maxIters = 10*length(y);
end

% Initialize
[d,n] = size(A); activationHist = []; iters = 0; res = y; sol =
zeros(n,1);

normb = norm(y); resnorm = normb; coef = [];

while (iters < maxIters) & (resnorm > NoiseLevel)

    [maxcorr i] = max(abs(A'*res));
    newIndex = i(1);

    % Project residual onto maximal vector
    Ai = A(:, newIndex);
    newCoeff = res' * Ai;
    res = res - (newCoeff ./ norm(Ai)) .* Ai;

    % Update solution
    sol(newIndex) = sol(newIndex) + newCoeff;
    activationHist = [activationHist newIndex];

    if verbose
        fprintf('Iteration %d: Adding variable %d\n', iters, newIndex);
    end
    iters = iters+1;

    resnorm = norm(res);

end

%
% Part of SparseLab Version:100
% Built:Sunday,18-Dec-2005 00:00:00
% This is Copyrighted Material
% For Copying permissions see COPYING.m
% Comments? e-mail SparseLab@stat.stanford.edu
%
\end{verbatim}

Notice that the source contains information about the author and
date of compilation, as well as copyright, of the routine.  Also,
the help information is built in as the first thing following the
function header.

\subsubsection{Documentation Directory}

The \WaveLab\ system also has extensive built-in documentation about the system
itself.  If you look in the directory {\tt Documentation}, you will find several
files of general interest:

\begin{verbatim}
% ADDINGNEWFEATURES       -  How to Add New Features to SparseLab
% BUGREPORT               -  How to report bugs about SparseLab
% COPYING                 -  SparseLab Copying Permissions
% DATASTRUCTURES          -  Basic data structures in SparseLab
% FEEDBACK                -  Give feedback about SparseLab
% GETTINGSTARTED          -  Ideas for getting started with SparseLab
% INSTALLATION            -  Installation of SparseLab
% LIMITATIONS             -  SparseLab known limitations
% PAYMENT                 -  No Charge for SparseLab Software
% REGISTRATION            -  SparseLab Registration
% SUPPORT                 -  SparseLab Support
% THANKS                  -  Thanks to contributors
% VERSION                 -  Part of SparseLab Version v090
% WARRANTY                -  No Warranty on SparseLab software
%
%              Subdirectories
%
%   AboutSparseLab        -  documentation for AboutSparseLab document
%   SparseLaArchitecture  -  documentation for SparseLabArchitecture document

\end{verbatim}



\subsubsection{Dataset Documentation}

Datasets are also documented. If you look in the directory {\tt
Datasets}, you will find that each dataset (\dotraw\ or \dotasc) is
accompanied by a \dotdoc\ file.
%The dataset {\tt daubechies.raw} is
%accompanied by the file {\tt daubechies.doc}, which contains the
%following:

%\begin{verbatim}
%daubechies.raw -- Gray-scale image of Ingrid Daubechies

%Access
%   Ingrid = ReadImage('Daubechies');

%Size
%   256 by 256

%Gray Levels
%   256

%Description
%   Ingrid Daubechies is a very active researcher in the field of
%   wavelet analysis and author of the book "Ten Lectures on Wavelets",
%   SIAM, 1992.  She is inventor of smooth orthonormal wavelets
%   of compact support.

%Source
%   Photograph of Ingrid Daubechies at the 1993 AMS winter meetings
%   in San Antonio, Texas. Taken by David Donoho with Canon XapShot
%   video still frame camera.
%\end{verbatim}

\subsection{Demos}

After browsing around to see what files \WaveLab\ contains,
it's time to see what \WaveLab\ can do!

The subdirectory {\tt \WLName /Papers} itself contains several subdirectories; each one of these
contains scripts that were used to produce figures in our published articles.

As new articles are written by members of our group, we will add new subdirectories.

Each subdirectory contains a ``demo'' file, such as {\tt ExtCSDemo.m}, which allows you to
reproduce the figures in the corresponding article.

When you invoke that file in \Matlab\ by typing its name (without the \dotm\ extension), a new
window will appear on the screen.  If you mouse-click on the push button {\tt Show All Figures} you
will see, in sequence, each figure in the corresponding article.  As each figure appears in
{\Matlab}'s figure window, the command window will contain narrative explaining what you see in the
figure window.


\subsubsection{Demo Inventory}

Here is an up-to-date listing of demos in version \WLVersion,
and the articles to which they correspond:
\begin{verbatim}
   ExtCSDemo    - demo for paper "Extensions of Compressed Sensing"
   HDCPNPDDemo  - demo for paper "High-Dimensional Centrosymmetric Polytopes with
                  Neighborliness Proportional to Dimension"
   MSNVENODemo  - demo for paper "Breakdown Point of Model Selection When the
                  Number of Variables Exceeds the Number of Observations"
   NPSSULEDemo  - demo for paper "Neighborly Polytopes and Sparse Solutions of
                  Underdetermined Linear Equations"
   NRPSHDDemo   - demo for paper "Neighborliness of Randomly-Projected Simplices
                  in High Dimensions"
   SNSULELPDemo - demo for paper "Sparse Nonnegative Solutions of Underdetermined
                  Linear Equations by Linear Programming"
   StOMPDemo    - demo for paper "Sparse Solution of Underdetermined Linear Equations
                  by Stagewise Orthogonal Matching Pursuit"
\end{verbatim}


\subsection{Examples}

We include a number of pedagogical examples in SparseLab, so that
the user can familiarize himself or herself with the software and
with our intentions in providing it. Currently we include:
\begin{verbatim}
   nnfEx             Nonnegative Factorization
   reconstructionEx  Signal Reconstruction
   RegEx             Model Selection in Regression
   TFDecompEx        Time Frequency Decomposition
\end{verbatim}

Each example is documented on the SparseLab website and can be run by running the correspondingly
named .m file in each directory. All scripts are documented to allow users to modify parameters
such as noise level, number of variables or observations, algorithm used, etc.

\subsection{Reproducible Research}

Jon Claerbout, a distinguished exploration geophysicist at Stanford, has in recent years championed
the concept of {\it really reproducible research} in the ``Computational Sciences.''

The ``Computational Sciences'' he has in mind are fields in which mathematical and computer science
heuristics may suggest algorithms to be tried on scientific signal processing and imaging problems,
but mathematical analysis alone is not able to predict fully the behavior and suitability of
algorithms for specific datasets. Therefore experiments are necessary and such experiments ought,
in principle, to be reproducible, just as experiments in other fields of science.

Some background information may help the reader.  Suppose we are working in an area like
exploration seismology where the goal is an image of the subsurface, and computational science
aims to produce better images. However, the deliverable is not an image itself, but instead the
software environment that, applied in the right way, produces the image, and which, hopefully,
could be applied to other datasets to produce equally nice images. The scientific findings may turn
out to be a {\it knowledge of parameter settings} for this complex software environment that seem
to lead to good results on real datasets.

With this as background, reproducibility of experiments requires  having the complete software
environment available in other  laboratories and the full source code available for inspection,
modification, and application under varied parameter settings.

Reducing this to a slogan: {\it An article about computational science in a scientific publication
is {\bf not} the scholarship itself, it is merely {\bf advertising} of the scholarship.  The actual
scholarship is the complete software development environment and the complete set of instructions
which generated the figures.}

The advantage of reproducibility for the progress of the discipline is clear. When a really good
idea is found, everyone else can be using it right away. When a  mistaken finding is reported, it
is rooted out almost immediately, etc.

The barriers to sharing complete software environments are also clear: if you are the developer of
a nice piece of software, you may not want to give other people the benefit of your investment of
time simply for the benefit of an abstract principle such as scientific progress. Even if you are
altruistic enough to  make your work available to others in this way, it's a lot of extra work to
generate code clean enough for others to look at; most people prefer to make the figures for their
articles using quick-and-dirty undocumented code, and move on to the next project.

Claerbout and his colleagues have developed a discipline for building their own software, so that
from the start, they expect it to be made available to others as part of the publication of their
work.  Specifically, they publish CD-ROMs (available from Stanford University Press) which contain
the text of their books along with a special viewer that makes those books {\it interactive
documents}, where as one reads the document, each figure is accompanied by the possibility of
pop-up windows which allow one to interact with the code that generated the figure, to ``burn'' the
illustration (i.e., erase the postscript  file supplied with the distribution),  and to rebuild the
figure from scratch on one's own machine.  By following the discipline of planning to  publish in
this way from the beginning, they maintain all their work  in a form which is easy to make
available to others at any point in time.

Why do we think this concept is significant for the research community? It is our perception that
as we approach specific applications using sparse solvers, we are becoming a computational science
like seismic imaging. In this setting, publishing figures or results without the complete software
environment could be compared to publishing an announcement of a mathematical theorem without
giving the proof.

With the recent rapid spread of Internet facilities worldwide and the standardization of scientific
computing on about five machine architectures, most of which are UNIX-based, it becomes feasible
and timely to actually experiment with protocols implementing the goal of reproducible research. We
may forget that in other sciences, the canonical form for scientific articles was arrived at
incrementally. For example, it has been said  that Pasteur introduced the notion of publishing a
full description of methods, materials and laboratory procedures.

The publication of \WaveLab\ is a modest step in this direction. Partly to indicate our esteem for
the work of Claerbout and the Stanford Exploration Project, we are making available code which will
allow the interested Internet-nik to reproduce the figures in our articles and to study the exact
parameter settings and algorithms which were used in those articles. This arrangement does not, at
the moment, conform to the Stanford Exploration Project's idea of interactive document, in which a
special TeX viewer is tied to special code resources that can rebuild and vary the figures in a
paper. However, at a primitive level, it provides the interested researcher with the TeX files of
the documents and everything necessary (sans \Matlab) to rebuild and modify the figures in those
articles.

We hope to publish code in this way for several forthcoming articles and also to learn more about
the concepts of {\it Interactive Document} and {reproducible research}.

\subsection{Freeware}

Richard Stallman and others associated with the GNU project have pioneered the idea of free
software -- software that can freely be redistributed by users. (This does not mean free of cost;
it means the rights any one person has over the software are the same as those of any other).  We
have been influenced by this, but obviously cannot completely follow them since we require of users
that they have \Matlab, which is not freely redistributable. In fact, Richard Stallman told one of
us that making our software available in \Matlab\ was the worst thing we could do, as this might
encourage people to buy \Matlab!

In our opinion, the Freeware concept is useful and has had a major impact. However, the Freeware
concept has limits, and the \Matlab\ example shows this clearly. Modern scientific computing
depends on {\it quantitative programming environments} like \Matlab, {\sc SPlus}, {\sc R}, {\sc
Mathematica}, {\sc X-Math}, {\sc IDL} and so on. These are widely available, widely understood
high-level languages in which key concepts of scientific discourse (Fast Fourier Transform, etc.)
are available as built-in, easily usable features.  At the moment, a complete Freeware
implementation of one's computational experiments requires, more or less, emulating Jon Claerbout's
example and writing all one's tools from scratch in C or Fortran.  Claerbout is forced to do this
because of the massive size of the datasets he uses, which exceed the bounds of \Matlab\ or other
QPE's. Most working scientists have smaller-scale datasets than Claerbout, and so they can use
modern QPE's. Moreover they are busy, and view scientific computing only as a sideshow.  They
cannot be expected to start from scratch and develop all their code in C when they can get, much
more quickly and reliably, the same results in a very high-level language. Thus the temptation to
work in a QPE is almost irresistible.

Stallman might retort that there are now freeware QPE's.  Octave is a \Matlab\ work-alike in many
ways, developed by John Eaton at the University of Texas Chemical Engineering department.  Octave
is developed strictly within the GNU philosophy, and so makes available a Stallman-acceptable QPE.
In the latest release, version 1.1, it comes close to \Matlab\ 4.X in many key ways, so that a
large fraction of what we do will run under Octave.  Perhaps one day an Octave port of \WaveLab\
will be feasible.

But where does a working scientist spend his/her effort? On ports to noncommercial systems that
will satisfy the urges towards binary liberation? Perhaps, but only with a lower priority than
other activities. It is our impression that, at the moment, there are relatively few scientists who
would complain about lack of access to \WaveLab\ due to the cost of \Matlab, and far fewer who
would complain that we fall short of full reproducibility by our dependence on a commercial tool at
one stage. Sorry to let you down, Richard.

\pagebreak

\section{Fine Print}

In making available our software, the authors have tried to follow some simple
guidelines.  We spell them out in this section  to avoid misunderstandings about
what we are offering, why we are offering it, what rights we give you and what
rights we retain for ourselves.

The directory {\tt SparseLab/Documentation} contains the following
files:
\begin{verbatim}
ADDINGNEWFEATURES       -  How to Add New Features to SparseLab

BUGREPORT               -  How to report bugs about SparseLab

COPYING -  SparseLab Copying Permissions

DATASTRUCTURES          - Basic data structures in SparseLab

FEEDBACK                -  Give feedback about SparseLab

GETTINGSTARTED          -  Ideas for getting started with SparseLab

INSTALLATION            - Installation of SparseLab

LIMITATIONS -  SparseLab known limitations

PAYMENT -  No Charge for SparseLab Software

REGISTRATION - SparseLab Registration

SUPPORT - SparseLab Support

THANKS - Thanks to contributors

VERSION -  Part of SparseLab Version v$VERSION$

WARRANTY - No Warranty on SparseLab software
\end{verbatim}
These describe the philosophy and limitations of our package.  The key points
are summarized here, by reprinting the contents of some of those files.

\subsection{Dependence on \Matlab}

\Matlab\ is a product of The Mathworks, a successful company based in Natick, Massachusetts. Their
product is expensive. You need it to run our software. We do not offer it. You must get it from
them. You need version 6.X or later. We have no connection with them. They have no connection with
us.

\subsection{Registration -- \WaveLab\ Registration}

(from file {\tt REGISTRATION.m})

Please Register yourself as a user of \WaveLab\ so that we can send you e-mail about upgrades and
enhancements. To Register: go to the Download page of the Sparselab website \WLWEB and click on the
link \underline{REGISTER}. If you like, please include information about the version of \Matlab\
you are using and about the type of machine you are using.

\subsection{Limitations}

This package has been designed to reproduce the figures in our research. Accordingly, it may not
solve the problems you have in mind. Please see the file {\tt LIMITATIONS.m} before complaining.
Perhaps we are already aware of the problem you have and are seeking to fix it!

\subsection{Support}

(from file {\tt SUPPORT.m})

This software has been developed as part of the research effort of the authors under various
federally supported grants.  If you find that it does not work correctly, please e-mail a
notification of your problem to \WaveLab.  Use the format indicated in the file {\tt BUGREPORTS.m}.

To the extent that we can isolate the problem and develop a solution, and to the extent that it
fits in with our schedule with releasing a new version, we will attempt to fix the problem.

\subsection{No Charge -- No Charge for \WaveLab\ Software}

(From file \Payment)

\WaveLab\ software is available at NO CHARGE from the SparseLab website at\\
\centerline{\WLWEB.} The software is copyrighted. For permissions to copy, see the file \Copying\
or you may e-mail \eWaveLab. If you use this software to produce scientific articles, we would
appreciate being informed of the article's title, authors, topic, and place of appearance. If this
software played a major enabling role in your scientific work, and you are so moved, you might
acknowledge us in your article.

\subsection{No Warranty -- No Warranty on SparseLab software}

(From file \Warranty)

There is no warranty attached to the \WaveLab\ software.

If you find that it does not work correctly, use the file BUGREPORTS.m as a template to compile a
description of the problem, including as far as possible a complete m-file script that generates
the error. E-mail the description to: \eWaveLab. Resources permitting, an effort will be made to
correct the problem for a future release.

\vskip .1in
\centerline{\bf FORMAL LEGAL DISCLAIMER OF WARRANTY}

We make no warranties, explicit or implicit, that the programs contained in this collection are
free of error, or are consistent with any particular standard of accuracy, or that they will meet
your requirements for any particular application. They should not be relied on for any purpose
where incorrect results could result in loss of property or personal injury. If you do use programs
for any such purpose it is at your own risk. The authors disclaim all liability of any kind, either
direct or consequential, resulting from your use of these programs.

\subsection{Copyright -- \WaveLab\ Copying Permissions}

(From file \Copying)

This software was developed with partial support of the National Science Foundation Grant DMS
92-07266. Contributors to this effort include David Donoho, Iddo Drori, Michael Saunders, Victoria
Stodden, Joshua Sweetkind-Singer, and Yaakov Tsaig.

In EVERY case, the software is COPYRIGHTED by the original author. For permissions, e-mail
\eWaveLab.

\vskip .1in
\centerline{\bf COPYLEFT NOTICE}

Permission is granted to make and distribute verbatim copies of this entire software package,
provided that all files, directories, and subdirectories are copied {\it together} as a {\it unit}.

Here {\it copying as a unit} means: all the files listed in the file {\tt SLFiles} are copied, so
that among other things  the \Thanks\ notice, and this permission notice accompany all copies of
the full software, and every ``\dotm'' file continues to contain a copy of the file {\tt VERSION.m}
inside it.

Permission is granted to make and distribute modified copies of this software, under the conditions
for verbatim copying, provided that the entire resulting derived work is distributed under the
terms of a permission notice identical to this one. Names of new authors, their affiliations and
information about their improvements may be added to the files \Copying\ and \Thanks.

The purpose of this permission notice is to allow you to make copies of the software and distribute
them to others subject to the constraint that you maintain the collection of tools here as a unit.
This enables people to whom you give the software to be aware of its origins, to ask questions  of
us by e-mail, to request improvements, obtain later releases, etc.

If you seek permission to copy and distribute translations of this software into another language,
please e-mail a specific request to \eWaveLab. If you seek permission to excerpt a {\it part} of
the software library  for example to appear in a scientific publication, please e-mail a specific
request to \eWaveLab.


\subsection{Thanks -- Thanks to contributors}

(from file \Thanks)

This software was developed with support of the National Science
Foundation grant DMS 00-07266. Contributors to this effort include
David Donoho, Iddo Drori, Michael Saunders, Victoria Stodden, Joshua
Sweetkind-Singer, and Yaakov Tsaig.

Further contributions are welcome, for info e-mail \eWaveLab.

%{\bf Contributors of Data}: The following have contributed data
%hich is part of our system. Evan Hurowitz, University of California San Francisco Department of
%Biopharmaceutical Sciences.

\input References.tex

\end{document}
